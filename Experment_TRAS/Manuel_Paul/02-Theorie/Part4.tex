%\chapter{Methods and Material}
%\label{chap:methods}

\section{Experimental Setup for wavelength dependent TRAS}
\label{sec:setup}
To investigate the wavelength dependence of the triplet population, either a wavelength variable probe laser can be used (serial) or by spectrally resolved detection of a white light probe pulse (parallel).

\section{Effect of the used Lens in the Beam Path}
\label{sec:effect}
If the lens with 60 mm focal length were exchanged for a lens with 75 mm focal length, the focus would no longer be in the sample, but behind it. This would result in a massive signal degradation and a significant increase in noise.

\section{Extend Shelf Life of the Solutions}
\label{sec:lifetime}
If the sample is in contact with oxygen, oxidation of the chromophoric groups can occur, this is called oxidative bleaching. Furthermore, these groups can also be destroyed by the effects of electromagnetic radiation. 

Airtight storage, possibly under inert gas, can prevent oxidation. In addition, dark storage can prevent bleaching by radiation.

\section{Concentration of a Polymer}
\label{sec:concentration}
The concentration corresponds to the specification if the corresponding monomers were used.


\section{Positive/Negative transient absorption}
Stimulated absorption due to the excitation pulse may fall within the range of the probe pulse, thus being detected and yielding a negative contribution to the transient absorption.

The probe pulse can be absorbed by molecules in the appropriate excited state. This causes a decrease in transmission and thus provides a positive contribution.