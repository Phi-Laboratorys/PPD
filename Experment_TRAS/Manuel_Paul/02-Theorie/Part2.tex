\newpage
\section{Kinetics of Triplet State Decay}
\label{sec:decay}

The kinetics of a triplet state decay can be described by the following differential equation
\begin{gather}
    -\frac{\mathrm{d}}{\mathrm{d}t} C_\mathrm{T} = k_1 C_\mathrm{T}(t) + k_2(C_\mathrm{T}(t))^2 + k_3C_\mathrm{T}(t)C_\mathrm{G}(t)~,
    \label{eq:kinetics}
\end{gather}
with $C_\mathrm{T}$ as concentration of the triplet state and $C_\mathrm{G}$ as concentration of the ground state. The relaxation in the ground state is decribed by the reaction rate $k_1$ and varios bimolecular deactivation processes by the rates $k_2$ and $k_3$. Assume that the concentration of the triplet state is $C_\mathrm{T}(t)$ is smaller than the concentration of the ground state $C_\mathrm{G}(t)$, because of the pump laser which excites the electrons into the singlet state and the process of intersystem crossing. Additionally assume that $C_\mathrm{G}(t) = C_\mathrm{G} = \mathrm{const}$, since the pump there is a finite number of singlet states. All this assumptions lead to the expression
\begin{gather}
    \Rightarrow -\frac{\mathrm{d}}{\mathrm{d}t} C_\mathrm{T} = (k_1  + k_3C_\mathrm{G})C_\mathrm{T}(t) = k_0 C_\mathrm{T}(t)~.
    \label{eq:kineticsReduced}
\end{gather}
Solving the differential equation for an given initial condition for the concentration of the triplet state $C_\mathrm{T}(0)$ results in 
\begin{gather}
    \Rightarrow \boxed{C_\mathrm{T} = C_\mathrm{T}(0) e^{-k_0t}}~.\cite{Kilchert.04.2023}
    \label{eq:kineticsSolved}
\end{gather}