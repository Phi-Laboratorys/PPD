% Teilaufgabe 1

\section{Beispiel für technische Anwendung der Plasmonik}

Die Plasmonik wird zum Beispiel im Bereich der elektrooptischen Modulatoren in der Kommunikationstechnik verwendet. Hierbei wird ein Phasenmodulator verwendet, um ein Laserlicht ein- und auszuschalten, womit elektronische Informationen von Einsen und Nullen auf ein optisches Signal encodiert werden können. Dies wird bewerkstelligt, indem ein elektrisches Feld an das Material im Phasenmodulator angelegt wird, was zur Folge hat, dass sich der Berechungsindex des Materials ändert. Die Berechungsindexänderung wiederum ändert die Phase des Lichts. Zusätzlich lässt sich mit einem interferometrischen Aufbau mit konstruktiver und destruktiver Interferenz eine Amplituden- bzw. Intensitätsmodulation des Lichts erzeugen. Zusammen lässt sich so des Laserlichts mithilfe der Plasmonik encodieren. \cite{Anwendung} 