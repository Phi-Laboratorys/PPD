% 1. Einleitung

\chapter{Einleitung}
\label{chap:einleitung}

Nanoobjekte, welche aus einigen hundert bis einigen hunderttausend Atomen bestehen,
bilden die Brücke zwischen rein atomarem und makroskopischem Verhalten. Nanosysteme weisen hierbei auf mesoskopischen Größenskala außergewöhnliche Eigenschaften auf, die sie zu fundamentalen Objekten in Grundlagenforschung und neuen, richtungsweisenden Technologien macht.

Plasmonen wiederum sind die kohärente Oszillation der Leitungsbandelektronen in metallischen Strukturen und ermöglichen eine kontrollierte Licht-Materie Wechselwirkung durch künstlich hergestellten optischen Strukturen und Materialien. Ein Beispiel davon sind sogenannte Metamaterialien, mit deren Hilfe Objekte sogar
unsichtbar gemacht werden können, indem das Licht um das Objekt herum geleitet wird,
ohne dabei im klassischen Sinne gestreut oder absorbiert zu werden. 

Der Versuch umfasst den Aufbau eines Dunkelfeldmikroskops, den Vergleich dieser Me-
thode mit der Hellfeldtechnik, und anschliessend die Verwendung des Mikroskops zur
Spektroskopie von Silber-Nanopartikeln. \cite{Anleitung}