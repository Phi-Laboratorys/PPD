% 1. Einleitung

\chapter{Einleitung}
\label{chap:einleitung}

Dieser Versuch nutzt Interferenzeffekte verusacht von einer Halogen-Deuterium-Lampe (Weißlicht), um optische Eigenschaften von lichtdurchlässigen Materialien zu untersuchen. Die Dicke dieser sog. dünnen Filme reicht dabei von einigen Nano- bis Mikrometern und können durch Reflexion als auch durch Transmission gemessen werden.

Mit diesen Messmethoden lassen sich der Brechungsindex, die Schichtdicke des Films und die Oberflächenbeschaffenheit (Homogenität) bestimmen. Die vergleichsweise simple, nicht destruktive Art der Messung und der geringe Aufwand der Probenpräparation machen diese Messmethode für Laboranwendungen besonders attraktiv und auch für diesen Praktikums Versuch.

Wir betrachten hierbei mehrere Proben mit unterschiedlichen Konzentrationen an Polystyrol und Chlorbenzol aufgebracht auf Glas (Substart) via Spincoating und Vergleichen die jeweilgen Messmethoden.