% Teilaufgabe X

\newpage

\section{Zusammenhang zwischen Absorptionskoeffizienten und Brechungsindex}
\label{sec:absorpBrechung}

Der Absorptionskoeffizient $\alpha$ ist proportional zum Imaginärteil des Brechungsindexes $\kappa$.  Dieser Zusammenhang kann mit einer erzwungen elektromagnetischen Schwingung der Form:
\begin{gather}
    m \derivative[2]{x}{t} + b \derivative{x}{t} + Dx = q E_0e^{i\omega t}
    \label{eq:schwingung}
\end{gather}
mit Masse $m$, Ladung $q$, Reibungskonstante $b$ und Rückstellmoment $D$ erklärt werden. Durch Lösen der Gleichung \ref{eq:schwingung} mit dem Ansatz $x=x_0e^{i\omega t}$ und Einsetzen der Lösung in die Gleichung der Polarisation $P = Nqx = \epsilon_0(\epsilon-1)E$ erhält man mit dem Brechungsindex $n$ von nichtferromagnetischen Materialien ($n = \sqrt{\epsilon}$) erhält man:
\begin{gather}
    n^2 = 1 + \frac{Nq^2}{\epsilon_0 m (\omega_0^2 - \omega^2 + i\gamma\omega)}~,
    \label{eq:brechungsindex}
\end{gather}
wobei $\gamma = b/m$ und $\omega_0^2 = D/m$ ist. Der Brechungsindex kann dann in den komplexen Brechungsindex $n(\omega)$ umgeschrieben werden:
\begin{gather}
    n = n' + i\kappa;~~~~ n',\kappa \in \mathbb{R}~,
\end{gather}
wobei $\kappa$ der Extinktionskoeffizient ist. Im nächsten Schritt verwendet man eine elektromagnetische Welle der Form:
\begin{gather}
    E = E_0 e^{i(\omega t - Kz)}~\mathrm{mit}~K = 2\pi/\lambda~,
    \label{eq:efeld}
\end{gather}
welche durch das Medium mit dem Brechungsindex $n$ in $z$-Richtung mit der Wellenzahl $K$ läuft. Im Vakuum ist dabei $K = K_0$ und in der Materie $K_\mathrm{M} = nK_0 = n'K_0 -i\kappa K_0$. Setzt man $K_\mathrm{M}$ für $K$ in Gleichung \ref{eq:efeld} ein, erhält man:
\begin{gather}
    E = E_0e^{-K_0\kappa z}e^{i(\omega t -n'K_0z)} = E_0e^{-2\pi \kappa z/ \lambda}e^{iK_0(c_0 t -n'z)}~.
    \label{eq:efeldc}
\end{gather}
Laut Beerschen Absorptionsgesetz ist der Absorptionskoeffizient $\alpha$ für den Intensitätsverlauf gegeben durch:
\begin{gather}
    I = I_0 e^{-\alpha z}~.
    \label{eq:beer}
\end{gather}
Da Intensität proportional zum Quadrat der Amplitude ist, wird Gleichung \ref{eq:efeldc} quadriert und die Exponenten mit denen der Gleichung \ref{eq:beer} verglichen. Der Vergleich ergibt dann:
\begin{gather}
    \boxed{\alpha = 4 \pi \kappa / \lambda = 2K \kappa}~,
\end{gather}
was die zuvor angesprochenen Zusammenhang ergibt. \cite{DemtroederLaser1}