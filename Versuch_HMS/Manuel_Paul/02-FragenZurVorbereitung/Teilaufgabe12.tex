% Teilaufgabe X

\section{Herleitungen}
\label{sec:herleitung}

In diesem Kapitel sollen Formel (3.6) und (3.7) aus dem Skript zu diesen Versuch hergeleitet werden. Da das Skript selbst nicht direkt auf den Ursprung der Formeln eingeht, wird diese Herleitung nur kurz mit der Literatur \citenum{FMSpectro} hergeleitet und dem Skript abgearbeitet.

Aus Literatur \citenum{FMSpectro} wird entnommen, dass folgende Relation gilt:
\begin{gather}
    \delta_\mathrm{n} = \alpha_\mathrm{n} \frac{L}{2}~.
    \label{eq:absorp}
\end{gather}
Aus dem Skript entnehmen wir von die Gleichungen (1.1) und (1.2) und setzen sie gleich, was folgende Beziehung ergibt:
\begin{gather}
    \frac{I}{I_0} = e^{-\alpha L} = 10^{-\mathrm{OD}} \Leftrightarrow \alpha = \frac{\ln 10}{L} \mathrm{OD}~. \cite{anleitung}
    \label{eq:absorpkoeff}
\end{gather}
Setzt man nun nur noch Gleichung \ref{eq:absorpkoeff} in Gleichung \ref{eq:absorp} erhält man die gewünschte Beziehung:
\begin{gather}
    \boxed{\delta_\mathrm{n} = \frac{\ln 10}{2} \mathrm{OD}}~.
\end{gather}

Weiterhin wird aus der Literatur \citenum{FMSpectro} entnommen:
\begin{gather}
    \phi_\mathrm{n} = \eta_\mathrm{n} L \left( \frac{\omega_\mathrm{c} + n \omega_\mathrm{m}}{c} \right)~.
    \label{eq:phase}
\end{gather}
Zusammen mit der Überlegungen
\begin{gather}
    c = \lambda f = \lambda \frac{\omega}{2\pi} \Leftrightarrow \frac{2\pi}{\lambda} = \frac{\omega}{c}~\mathrm{mit}~\omega=\omega_\mathrm{c} + n \omega_\mathrm{m}
\end{gather}
muss dann eingesetzt in Gleichung \ref{eq:phase} gelten:
\begin{gather}
    \boxed{\phi_\mathrm{n} = \eta_\mathrm{n} L \left( \frac{2\pi}{\lambda} \right)}~,
\end{gather}
was die gesuchte Gleichung (3.7) aus dem Skript ergibt.