% 1. Einleitung

\chapter{Einleitung}
\label{chap:einleitung}

Optische Absorptionsmessungen zählen zu den Standardverfahren, die in vielen Bereichen der Physik, aber auch der Chemie und Biologie zur Charakterisierung von Substanzgemischen sowie zur Untersuchung physikalischer Eigenschaften von Atomen, Molekülen und Festkörpern eingesetzt werden.

In herkömmlichen Absorptionsexperimenten misst der Photodetektor bei Wellenlängen außerhalb der Absorptionslinien eine bestimmte Lichtstärke, die im Bereich der Linien abnimmt. Ein solches Messverfahren bezeichnet man als “nicht hintergrundfrei”, weil die Absorptionslinien als Einbrüche in einem Signalhintergrund erscheinen. Die Empfindlichkeit, d. h. der kleinste nachweisbare OD-Wert, wird hierbei durch die Schwankungen des Hintergrundsignales begrenzt, zu denen neben dem fundamentalen Quantenrauschen — insbesondere bei Verwendung eines Lasers — verschiedene “technische” Rauschquellen beitragen können. Absorptionslinien, die schwächer sind als ca. 10 bis 3 in Einheiten der optischen Dichte, lassen sich damit nur schwer nachweisen.

Um schwächere Absorptionssignale messen zu können, benötigt man ein Messverfahren ohne Einflüsse des Hintergrundes, das nur innerhalb einer Absorptionslinie ein Signal liefert. Eine solche Methode ist die \text{optische Frequenzmodulationsspektroskopie} (optische FM-Spektroskopie). \cite{anleitung}