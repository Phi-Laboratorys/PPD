% Teilauswertung X

\newpage
\section{Absorption und Dispersion einer Jodlinie}


\begin{figure}[h]
    \centering
    \includegraphics[width=0.75\textwidth]{Bilder/Jodlinie/Gruppe32020Iododgraph1 Kopie.png}
    \caption[Iodlinie]{Iodlinie (Ersatzmessung, Gruppe 3, 2020)}
\end{figure}

%\subsection{Nachprüfen der Absorption}

Zur Nachprüfung des Berechneten Absorptionssignals wurden bei drei Frequenzen manuell eine Messung mit und ohne Probe durchgeführt. $\Delta U$ ist die Differenz der Spannung von Basis zur Spitze, jeweils durch die Probe und ohne die selbige. Aus diesen Werten wird dann die optische Dichte berechnet.


\begin{table}[h]
    \centering
    \begin{tabular}{c|cc|cc}
        $\omega_m$/MHz & $\Delta U_{Ref}$ / mV & $\Delta U_{Probe}$ / mV & OD/2,22 & OD \\ \hline
        79,7 & $128 \pm 4$ & $130 \pm 4$ & $0,00 \pm 0,02$ & $0,00 \pm 0,04$ \\
        491,9 & $268 \pm 4$ & $270 \pm 4$ & $0,003 \pm 0,009$ & $0,007 \pm 0,02$ \\
        989,1 & $283 \pm 4$ & $320 \pm 4$ & $0,053 \pm 0,008$ & $0,12 \pm 0,02$ \\
    \end{tabular}
    \caption{Ergebnisse für manuelle Messung der optischen Dichte.}
\end{table}

Betrachtet man die Werte aus der obigen Tabelle, für die optische Dichte, so fällt auf, dass diese mit der Frequenz ansteigen. Von 0 bei 79,7 MHz bis ca. 0,05 bei 989,1 MHz. Vergleicht man dies mit der obigen Abbildung der Absorptionslinie für Iod, so stimmen diese im Trend gut überein. Die Letzten zwei Werte weichen jedoch etwas von denen aus der Abbildung ab. Hierzu ist anzumerken, dass die Abbildung von einer Anderen Gruppe aus dem Jahr 2020 verwendet wurde, da der Versuchsaufbau nicht einwandfrei funktioniert hat. In wie weit die Messungen also vergleichbar sind ist unklar.

Hier wurde mit einem Korrekturfaktor von 2,22 gearbeitet, da unsere Rechteckfunktion aber eher Trabetzförmig ist und die Ecken etwas abgerundet sind, weicht der Wert des Faktors wohl etwas nach unten ab.

Für die Vakuumwellenlänge der Iodlinie gilt:
\begin{gather}
    \lambda = \frac{c_0}{f} = (632,9915 \pm 0,0004) \, \text{nm}
\end{gather}
In Luft ($n_{Luft} = 1,0003$) gilt:
\begin{gather}
    \lambda_{Luft} = \frac{\lambda}{n_{Luft}} = (632,8017 \pm 0,0004)\, \text{nm}
\end{gather}

Aus dem Diagramm konnte Breite der Jodlinie, am halben Maximalwert, von ca. 848MHz abgelesen werden.
Für die Dopplerbreite gilt:
\begin{gather}
    \delta f = \frac{f_0}{c} \sqrt{\frac{8 k_B T ln2}{m}}
\end{gather}
In unserem Fall git: $f_0 \approx 880$ MHz,  $T \approx 294$ K und $m = 2 \cdot 126,9 u$.
\begin{gather}
    \Rightarrow \delta f \approx 678 \, \text{MHz}
\end{gather}
Es fällt auf, dass der ermittelte Wert merklich größer ist als der theoretische. Dies liegt daran, dass es eine nicht aufgelöste Hyperfeinstruktur gibt, deren Komponenten ebenfalls eine Dopplerverbreiterung erfahren.


Im folgendem soll aus de Extremwerten der Phasenändeerung $\Delta \Phi$ die Brechungsindexänderung $\Delta \eta$ berechnet werden. Folgende Formel wird dafür genutzt:
\begin{gather}
    \Delta \eta = \Delta \Phi \frac{\lambda}{2 \pi L}
\end{gather}
Es gilt: $\Delta \Phi \approx 0,11$. Daraus folgt:
\begin{gather}
    \Delta \eta \approx 2,8 \cdot 10^{-8}
\end{gather}

Betrachtet man den Nullldurchgang des Dispersionssignales so fällt auf, dass er vor dem Peak des Absorptionssignales liegt. Bei gültigen Kramer-Kronig-Realationen würde man erwarten das sich diese Punklte übereinander befinden. Dies kann möglicherweiße durch eine Leistungsverbreiterung der Linien verursacht worden sein. Diese würde dazu führen, dass die mathematische Voraussetzungen für die Gültigkeit der Relationen nicht erfüllt wären.
Auseerdem ist der Verlauf der der Absorptionslinie bezüglich des Nulldurchgangs auffallend, bei gültigen Krameers-Kronig-Relationen würde man einen symmetrischen Verlauf erwarten.

Dispersionssignal der Etalonresonanz??