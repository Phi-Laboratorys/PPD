% Teilaufgabe X

\section{Physikalische Eigenschaft der Hyperfeinstruktur von Jod und Brom}
\label{sec:hyperfeinstruktur}

Die Hyperfeinstruktur impliziert die physikalische Eigenschaft, dass die Atomkerne ein magnetisches Moment besitzen. Dabei wird das magnetische Moment in der Quantenmechanik in der Form:
\begin{gather}
    \abs{\vect{I}} = \hbar \sqrt{I\cdot (I+1)}
\end{gather}
beschrieben. \cite{DemtroederKerne} Die Aufspaltung der Hyperfeinstruktur liefern dann zwei Beiträge:
\begin{itemize}
    \item Wechselwirkung des Kernmomentes mit dem Magnetfeld, das von Elektronen am Kernort erzeugt wird (Zeeman-Effekt des Kernmomentes mit dem atomaren Magnetfeld) \cite{DemtroederAtome}
    \item Wechselwirkung des elektronischen magnetischen Moment mit dem vom Kernmoment erzeugten Magnetfeld \cite{DemtroederAtome}
\end{itemize}