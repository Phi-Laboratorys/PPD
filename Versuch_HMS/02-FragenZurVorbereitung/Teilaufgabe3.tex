% Teilaufgabe X

\section{Berechnung der Signalleistung einer schmalen Absorptionslinie}

Für das gemessene Signal mit einem Messgerät mit 50 $\Omega$ Impedanz gilt:
\begin{gather}
    i_{S,mess}^2 = \frac{1}{2}(\frac{e\beta_D}{2hc}\lambda P M \frac{ln10}{2}OD)^2
\end{gather}

Für die Leistung gilt:
\begin{align}
    P_{S,mess} &= \frac{1}{\sqrt{2}}\frac{\beta_D}{2} P M \frac{ln10}{2} OD\\
    P_{S,mess} &= \frac{1}{\sqrt{2}}\frac{0,3}{2} \cdot 0,1 \, \text{mW} \cdot 0,4 \cdot \frac{ln10}{2} \cdot 0,1 = 5 \cdot 10^{-7} \, \text{W} = -33 \, \text{dBm}
\end{align}

Unter Berücksichtigung der Verstärkung gilt:
\begin{gather}
    P_{S,v} = -33 \, \text{dBm} + 55 \, \text{dB} = 22 \, \text{dBm}
\end{gather}

Somit ergibt sich ein Signal-Rausch-Abstand von $34$ dB.