% Teilaufgabe X

\section{Berechnung der Signalleistung einer schmalen Absorptionslinie}
\label{sec:signalAbsorp}

Für das gemessene Signal mit einem Messgerät mit 50 $\Omega$ Impedanz gilt:
\begin{gather}
    i_\mathrm{S,Mess}^2 = \frac{1}{2}\left(\frac{e\beta_\mathrm{D}}{2hc}\lambda P M \frac{\ln10}{2}\mathrm{OD}\right)^2
\end{gather}

Für die Leistung gilt:
\begin{align}
    \label{eq:leistungPhotosignal}
    P_\mathrm{S} &= \frac{1}{2}\left(\frac{e\beta_\mathrm{D}}{2hc}\lambda P M \frac{\ln10}{2}\mathrm{OD}\right)^2 \cdot R\\
    P_\mathrm{S} &= \frac{1}{2} \left( \frac{e\cdot 0,3}{2\cdot h c} \cdot (632,8 \cdot 10^{-9}) \, \mathrm{m} \cdot (0,1 \cdot 10^{-3}) \, \mathrm{W} \cdot 0,4 \cdot \frac{ln(10)}{2} \cdot 0,1\right)^2 \cdot 50 \Omega \\
    P_\mathrm{S} &= 3,11 \cdot 10^{-9}\,\mathrm{mW} \qquad \Rightarrow \quad P_\mathrm{S}^* = -85 \, \mathrm{dBm}
\end{align}

Unter Berücksichtigung der Verstärkung (Verringerung muss nicht angewendet werden, da schon die angepasste Formel genutzt wurde) gilt:
\begin{gather}
    P_\mathrm{S,Theo}^* = -85 \, \text{dBm} + 55 \, \text{dBm} = -30 \, \text{dBm}
\end{gather}
Somit ergibt sich ein Signal-Rausch-Abstand von 49\,dB.