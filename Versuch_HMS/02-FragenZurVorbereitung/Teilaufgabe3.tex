% Teilaufgabe X

\section{Berechnung der Signalleistung einer schmalen Absorptionslinie}
\label{sec:signalAbsorp}

Für das gemessene Signal mit einem Messgerät mit 50 $\Omega$ Impedanz gilt:
\begin{gather}
    i_\mathrm{S,Mess}^2 = \frac{1}{2}(\frac{e\beta_\mathrm{D}}{2hc}\lambda P M \frac{\ln10}{2}\mathrm{OD})^2
\end{gather}

Für die Leistung gilt:
\begin{align}
    \label{eq:leistungPhotosignal}
    P_\mathrm{S,Mess} &= \frac{1}{\sqrt{2}}\frac{\beta_\mathrm{D}}{2} P M \frac{\ln10}{2} \mathrm{OD}\\
    P_\mathrm{S,Mess} &= \frac{1}{\sqrt{2}}\frac{0,3}{2} \cdot 0,1 \, \text{mW} \cdot 0,4 \cdot \frac{\ln10}{2} \cdot 0,1 = 5 \cdot 10^{-7} \, \text{W} = -33 \, \text{dBm}
\end{align}

Unter Berücksichtigung der Verstärkung gilt:
\begin{gather}
    P_\mathrm{S,Total} = -33 \, \text{dBm} + 55 \, \text{dB} = 22 \, \text{dBm}
\end{gather}

Somit ergibt sich ein Signal-Rausch-Abstand von $34$ dB.