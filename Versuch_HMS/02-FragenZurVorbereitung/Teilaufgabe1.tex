% Teilaufgabe 1

\section{Rauschquellen}
\label{sec:rauschen}

\paragraph*{Quantenrauschen}
Aufgrund der Quantisierung des Lichtes (in Photonen) ist es aus statistischen Gründen unmöglich, dass die Intensität der Lichtes absolut konstant ist. Es gilt folgender Zusammenhang für das mittlere Quadrat des Rauschstromes:
\begin{gather}
    i^2_\mathrm{Q} = 2ei_\mathrm{Ph}\Delta \nu = \frac{2e^2\beta_\mathrm{D}}{hc}\lambda P \Delta \nu
\end{gather}
wobei\\
\begin{tabular}{rl}
    $e$: & Elementarladung \\
    $i_\mathrm{Ph}$: & Photonenstrom des Detektors\\
    $\Delta \nu$: & Detektionsbandbreite\\
    $\beta_\mathrm{D}$: & Quantenausbeute (Nachweiswahrscheinlichkeit) des Detektors\\
    $\lambda$: & Wellenlänge\\
    $P$: & Lichtleistung\\
    $h$: & Planksches Wirkungsquantum\\
    $c$: & Lichtgeschwindigkeit
\end{tabular}\\
Es ist deutlich zu sehen, dass das Quantenrauschen mit der Quantenausbeute, der Wellenlänge, der Lichtleistung und der Detektionsbandbreite ansteigt.

\paragraph*{Thermisches Rauschen der Photodiode}
Dieses Rauschen ensteht durch die thermische Bewegung der Elektronen in der Photodiode. Die thermische Rauschleistung ergibt sich folgendermaßen:
\begin{gather}
    P_\mathrm{T} = 4 k_\mathrm{B} T \Delta \nu
\end{gather}
wobei $k_B$ der Boltzmannfaktor, $T$ die absolute Temperatur ist und $\Delta \nu$ die Detektionsbandbreite. Somit ist auch klar ersichtlich, dass dieses Rauschen mit der Temperatur und Detektionsbandbreite zunimmt.


\paragraph*{Technisches Laserrauschen}
Dieses Rauschen tritt Aufgrund der Technischen eigenheiten des Lasers auf. Beispielsweise kann dieses Rauschen durch Schwingungen der Laserspiegel zueinander oder Instabilitäten in der Gasentladung, bei Gaslasern, verursacht werden. Eine Besonderheit dieses Rauschens ist es, dass es nicht weiß ist, sondern es vor allem bei niedrigen Frequenzen auftritt. Oberhalb einiger MHz ist es dann vernachlässigbar. 
