% Teilaufgabe X

\section{Vergleich thermisches Rauschen und Quantenrauschen}

Im Folgenden soll die Leistung (in dBm) de thermischen Rauschens und des Quantenrauschens für folgenden Fall berechnet werden:

$P = 0,1$ mW, $T = 293$ K, $R_D = 50\, \Omega$, $\beta_D = 0,3$ und $\Delta \nu = 10$ kHz. Aufgrund der Impedanz erscheint der gemmessene hochfrequente Strom um den Faktor 2 kleiner.

Für die Leisungsschwankung, welch durch Quantenrauschen verursacht wird, gilt:
\begin{align}
    \Delta P_\mathrm{Q} &= \frac{1}{2} \sqrt{\frac{2\beta_\mathrm{D} h c}{\lambda}P \Delta \nu}\\
    \Delta P_\mathrm{Q} &= \frac{1}{2} \sqrt{\frac{2 \cdot 0,3\cdot h c}{632,8 \cdot 10^{-9}} \cdot 0,1\, \text{mW} \cdot 10 \, \text{kHz}} = 2,17 \cdot 10^{-10}\, \text{W} = -67\, \text{dBm}
\end{align}

Für das thermische Rauschen gilt:
\begin{gather}
    P_\mathrm{T} = 2 \cdot k_B \cdot 293\,\text{K} \cdot 10\,\text{kHz} = 8,1 \cdot 10^{-17}\, \text{W} = -131 \, \text{dBm}
\end{gather}

Es ist zu erkennen, dass das Thermische Rauschen gegenüber dem Quantenrauschen zu vernachlässigen ist.
Somit reicht es die Verstärkung auf das Quantenrauschen anzuwenden, um den Rauschpegel zu erhalten:
\begin{gather}
    P_\mathrm{Q,Total} = -67 \, \text{dBm} + 55\, \text{dB} = -12 \, \text{dBm}
\end{gather}