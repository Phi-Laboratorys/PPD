% 1. Introduction

\chapter{Introduction}
\label{chap:intro}

In the following, we will present our research on X-Ray diffraction, a versatile measurement method applied in various fields, including industrial applications, solid-state physics, and material science. One of the significant advantages of using X-Rays is their smaller wavelength, because of that, with X-Rays it is possible to operate beyond the limit of resolution of optical microscopy. This enables the representation of smaller structures with higher precision. However, constructing X-Ray lenses is a highly complex task due to the limited refractive index of most materials in this spectral range, which is typically close to or slightly below one. Consequently, the indirect method of "X-Ray Diffraction" is preferred.
\bigskip

In this experiment the X-Ray spectrum of different types of foil will be investigated to determine the type of material and the complexes NaCl (Sodium Chloride) and ZrSiO4 (Zircon/Zirconium(IV)Silicate) will be analyzed with the use of the diffraction method.